\chapter{Enviroments}
Many 
Since computer vision tasks are its own research domain, not yet fully explored, we conclude that tackling 
difficulties that would reliance on coputer vision model bring is not within the scope of this thesis. 
Consequently we focus mainly on enviroments with more semantically rich output.
\section{Gymnasium}

\subsection{Cartpole}
Cartpole is one of the most known benchmark enviroments in the Gymnasium library. It's a simple enviroment where the agent controls a cart rolling on flat surface with a pole on top.
The goal is to keep the pole balanced on top of the cart. The reward is calculated every step dending on the angle the stick has to the ground. The more unstable the angle the less the reward. 
When the angle goes over 30 degree deviation the enviroment episode is terminated.
\subsection{Lunar Lander}
Lunar Lander is somewhat more complex domain to optimise for than the Cartpole.  
Every step the reward is calculated based on several criteria, as oposed to only an angle of a stick, controling the landers descend and additional points are awarded for completing certain tasks.
Additionally there is a number of variables determining the enviroments behaviour during evaluaseveraltions such as the strenght of wind.
The observation space is also more complex, mixing continuous values like position and speed with boolean representations of the landers legs tpuching or not touching the ground.
